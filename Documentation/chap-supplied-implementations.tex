\chapter{Supplied implementations}
\label{chap-supplied-implementations}

\section{Standard line}
\label{sec-standard-line}

\subsection{Package}

The package named \texttt{cluffer-standard-line} is used for all names
that are specific to the implementation of standard lines.

\subsection{Classes}

\Defclass {line}

This class is a subclass of the protocol class named \texttt{line} in
the package named \texttt{cluffer}.

\Defclass {open-line}

This class is a subclass of the class named \texttt{line}.  It is used
when the contents of the line is modified.

For this class, the contents is represented as a simple \emph{gap
  buffer} so that adding or deleting items is done at the beginning or
the end of the gap.

\Defclass {closed-line}

This class is a subclass of the class named \texttt{line}.  It is used
when the contents of the line has not been modified after an
invocation of the function \texttt{items}.

For this class, the contents is represented by a simple \commonlisp{}
vector.  Whenever \texttt{items} is invoked on a closed line and the
entire contents is asked for, this simple vector is returned, and no
copy is made.

\section{Simple line}

\subsection{Package}

The package named \texttt{cluffer-simple-line} is used for all names
that are specific to the implementation of simple lines.

\subsection{Classes}

\Defclass {line}

This class is a subclass of the protocol class named \texttt{line} in
the package named \texttt{cluffer}.

Contrary to the standard line \seesec{sec-standard-line} the simple
line has no concept of open or closed lines.  All lines are
represented the same way with the items stored in a simple
\commonlisp{} vector.

\section{Standard buffer}

\section{Simple buffer}
