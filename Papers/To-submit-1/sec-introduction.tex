\section{Introduction}

Many applications and libraries contain a data structure for storing
and editing text.  In a simple input editor, it can be a single,
relatively short, line of text, whereas in a complete text editor,
texts with thousands of lines must be supported.

In terms of abstract data types, one can think of an editor buffer as
an \emph{editable sequence}.  The problem of finding a good data
structure for such a data type is made more interesting because a data
structure with optimal asymptotic worst-case complexity would be
considered as having too much overhead, both in terms of execution
time, and in terms of memory requirements.

For a text editor with advanced features such as keyboard macros, it
is crucial to distinguish between two different control loops:

\begin{itemize}
\item The innermost loop consists of inserting and deleting individual
  items (typically characters) in the buffer, and of moving one or
  more \emph{cursor} from one position to an adjacent position.
\item The outer loop consists of updating the \emph{views} into the
  buffer.  Each view is typically an interval of less than around one
  hundred lines of the buffer.
\end{itemize}

%%  LocalWords:  startup runtime
